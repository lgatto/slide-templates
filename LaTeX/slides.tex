\documentclass{beamer}

%-----------------------
%% This template is based on a template developed by the SMCS
%% (Statistical Methodology and Computing Service) at UCLouvain

%-----------------------
% pour faire les slides papiers :
% \documentclass[handout, compress]{beamer}
% \usepackage{pgfpages}
% % 4 par page
% \pgfpagesuselayout{4 on 1}[a4paper,border shrink=5mm, landscape]
% % 2 par page
% \pgfpagesuselayout{2 on 1}[a4paper,border shrink=5mm]

\usetheme{UCL2018}

\usepackage[utf8]{inputenc}
\usepackage[T1]{fontenc}
\usepackage{lmodern}
\usepackage{xspace}
\usepackage{float}
\usepackage{graphicx}
\usepackage{amssymb}
\usepackage{hyperref}

%% For slides in French
%% \usepackage[french]{babel}
%% \usepackage[cyr]{aeguill}

\usepackage{lipsum}

\theoremstyle{example}
\newtheorem{examplef}{Example}
\newtheorem{examplesf}{Examples}
\newcommand{\ei}{\end{itemize}}

\usepackage{xmpincl}


\title{Title of the presentation}
\author[\url{http://lgatto.github.io/about}]{Laurent Gatto}
\date{\today}
\institute[]{CBIO, de Duve Institute, UCLouvain}


\begin{document}

\pdfinfo {/Author(Laurent Gatto - UCLouvain)}


%-----------------------------------------------
% Title page
%-----------------------------------------------

\begin{frame}[plain]
\titlepage
\end{frame}


\begin{frame}%[noframenumbering]
%\thispagestyle{empty}

% Logo UCL a gauche
\begin{tikzpicture}
  \useasboundingbox (0,0) rectangle(\the\paperwidth,1);
  \node[inner sep=0pt] at (1.7,.5) {\includegraphics[width=.33\textwidth]{UCL_2018}};
 \end{tikzpicture}
\vspace{.1cm}

\textbf{Abstract:} \lipsum[1] 

\end{frame}


\begin{frame}{Frame title}

  \textcolor{SMCSblue}{Title}\newline
  Some sentence or small paragraph

  \bigskip

  \begin{block}{Block title}
    Block text block text block text block text block text block text
    block text.
  \end{block}

\end{frame}


%-----------------------------------------------
% Table of content
%-----------------------------------------------

\AtBeginSection[]{
  \setbeamercolor{background canvas}{bg=UCLblue2}
  \setbeamercolor{section in toc}{fg=UCLblue}
  \setbeamercolor{subsection in toc}{fg=UCLblue}
  \setbeamerfont{section in toc}{size=\large}

  \mode<handout>{
    \setbeamercolor{background canvas}{bg=white}
    \setbeamercolor{section in toc}{fg=UCLblue}
    \setbeamercolor{subsection in toc}{fg=UCLblue}
  }

  \begin{frame}[plain]
    \frametitle{Outlines}
    \tableofcontents[currentsection,hideothersubsections]
  \end{frame}

  \setbeamercolor{background canvas}{bg=white}
}

% \AtBeginSubsection[]
% {
% \begin{frame}
% \frametitle{Sommaire}
% %\tableofcontents[currentsection,currentsubsection,hideothersubsections]
% \tableofcontents[ 
% currentsection,
% currentsubsection, 
% hideothersubsections, 
% sectionstyle=show/shaded, 
% subsectionstyle=show/shaded/hide, 
% ] 
% \end{frame}
% }


%-----------------------------------------------
% Content
%-----------------------------------------------

\section{Introduction}

\begin{frame}{Introduction}

In the 1970s \textbf{S}, a language to program with data, is developed
at Bell Laboratories by a team managed by John M. Chambers.\newline

At the end of the 1980s \textbf{S-Plus}, a commercial
software.\newline

\textbf{R} was created in 1997 by Ross Ihaka and Robert Gentleman at
the Auckland University (New Zealand).\newline

\textbf{R} is a \textit{free} software (GNU Public Licence)\newline

Many statistical tools included in \textbf{R} and it is possible to
create more (thousands of packages made by users can be found on
internet)\newline

\url{https://cran.r-project.org/}

\end{frame}


%-----------------------------------------------
% New section
%-----------------------------------------------

\section{Vectors}

\subsection{Creating a vector}


\begin{frame}[fragile]{Concatenation}

  Remark : By default R displays 7 decimals. Here is the command to
  modify it: \texttt{options(digits=2)} (two decimals).

\end{frame}


\begin{frame}[fragile]{\texttt{seq} et \texttt{rep}}

  \vspace{-.2cm}
  \begin{itemize}
  \item \texttt{seq} allows to make a vector defined by a sequence:

  \item \texttt{rep} creates a vector which is the repetition of numbers or strings:
  \end{itemize}

\end{frame}


%-----------------------------------------------
% Final slide
%-----------------------------------------------

\begin{frame}%[noframenumbering]
%\thispagestyle{empty}

% Logo UCL a gauche
\begin{tikzpicture}
  \useasboundingbox (0,0) rectangle(\the\paperwidth,1);
  \node[inner sep=0pt] at (1.7,.5) {\includegraphics[width=.33\textwidth]{UCL_2018}};
 \end{tikzpicture}
\vspace{.1cm}


Acknowledgement:

\bigskip

Contact:

\begin{center}
  laurent.gatto@uclouvain.be – \url{lgatto.github.io/about}
\end{center}
 
\end{frame}


\end{document}
